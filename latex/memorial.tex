\documentclass[10pt,a4paper,sans,colorlinks]{moderncv}

% -- moderncv

% shared moderncv code

\moderncvstyle{banking}
\moderncvcolor{blue}
\name{Alan}{Guedes}
\address{London}{UK}
\email{a.guedes@ucl.ac.uk}
\homepage{alanlivio.github.io}
\social[linkedin]{alanlivio}
\social[twitter]{alanlivio}
\social[github]{alanlivio}
\social[googlescholar]{alanlivio}

% imports
\usepackage{footmisc} % enabling footnotes for moderncv
\usepackage{markdown} % import partials.md
\def\markdownLaTeXRendererDirectOrIndirectLink#1#2#3#4{%
\href{#3}{#1}{}{}}
\usepackage[american]{babel}
\usepackage{subcaption}
\usepackage{graphicx}
\usepackage{color}
\usepackage{soulutf8}
\usepackage{xcolor}
\usepackage{color,soulutf8}
\definecolor{darkgray}{RGB}{204,204,204}
\definecolor{gray}{RGB}{217,217,217}
\definecolor{lightgray}{RGB}{239,239,239}
% -- modercv has no figure
\newfloat{figure}{htbp}{figs}
% -- font
\usepackage{lmodern}
\usepackage[T1]{fontenc}
\usepackage[utf8]{inputenc}
\usepackage{textcomp}
\usepackage{amsfonts}
% -- lipsum
\usepackage{lipsum}
% -- tables
\usepackage{multirow}
\usepackage{makecell}
\usepackage{tabularx}
\usepackage{multirow}
\newcolumntype{C}[1]{>{\centering\arraybackslash}p{#1}}
% -- itemize
\usepackage{enumitem}
\SetEnumitemKey{mynosep}{noitemsep, nosep, topsep=0pt, partopsep=0pt, parsep=0pt, itemsep=0pt, leftmargin=*,rightmargin=0pt}
% -- url
\PassOptionsToPackage{hyphens}{url}
\usepackage{url}
% -- listing
\usepackage{listings}
% -- spacing
\usepackage{setspace}
% -- geometry funcs
\newcommand\hfGeometrySetLarge{
  \usepackage[a4paper, left=20mm, top=20mm]{geometry}
}
\newcommand\hfGeometrySetLargeNoBottom{
  \usepackage[a4paper, left=20mm, top=20mm, bottom=0cm]{geometry}
}
\newcommand\hfHyperrefSetupBlueLinks{
  % should be called inside document
  \definecolor{links}{HTML}{2A1B81}
  \hypersetup{
    colorlinks=true,
    urlcolor=links,
    linkcolor=links,
    citecolor=links
  }
}
% -- enum funcs
\newcommand\hfEnumItemHyphen{
  \usepackage{enumitem}
  \def\labelitemi{-}
}
\newcommand\hfParagraphReduceSpacing{
  \linespread{1}
  \setlength{\parskip}{0pt}
  \setlength{\parsep}{0pt}
}

\title{Memorial}
\nopagenumbers{}
\hfGeometrySetLarge

% -- \begin{document}

\begin{document}
\hfHyperrefSetupBlueLinks
\makecvtitle

\setlength{\parindent}{15pt}
I am \myname, Brazilian, currently living in Rio de Janeiro, where I act as Post-Doc at \href{http://telemidia.puc-rio.br/}{TeleMidia Lab}.
This document is my memorial.
I briefly summarize my professional career by describing roles and projects that I get involved in \href{http://www.lavid.ufpb.br}{Lavid Lab} at João Pessoa and \href{http://telemidia.puc-rio.br/}{TeleMidia Lab} at Rio de Janeiro.
Both labs are known for their expertise in video research and to be creators of the Ginga technology.
Ginga is a \href{https://www.itu.int/rec/T-REC-H.761}{international royalty-free standard} for interactive video applications in different TV contexts (Broadcast, Broadband, IPTV and IBB).
More than 13 countries in South America and Africa adopt it.

At \textbf{João Pessoa}, I acquired my Bachelor~(2009) and M.Sc.~(2012) degrees in Computer Science from the \href{www.ufpb.br}{UFPB}.
I started by acting as a developer at \href{https://www.dynavideo.com.br}{Dynavideo Embedded Systems}, working on a commercial implementation of Ginga on Digital TV receivers.
This period gave me skills in the development process and TV embedded hardware APIs.
Such experience was useful to work on Lavid, where I acted as a researcher assistant in
the \href{http://www.redetic.rnp.br/ctic/2019/01/29/gingarap-gingafrevo/}{Ginga CDN}
project, funded by the Brazilian \href{https://www.rnp.br}{ National Research Network~(RNP)}.
It aimed at developing open source software and creating a software community around Ginga and TV technologies in Brazil.
This project enables me to interact with different research groups in Brazil that study video and TV.
On my mastering, the professor \href{https://www.linkedin.com/in/guido-lemos-5361a48/?originalSubdomain=br}{Guido Lemos} advised me.
He aimed at improving Ginga on new contexts and allow me to lead a team on \href{http://www.redetic.rnp.br/ctic/2019/01/29/ginga-appstore/}{GingaStore} project, which is also funded by RNP.
This project developed an interactive portal prototype that delivers Ginga applications using broadcast and broadband networks.
My master dissertation describes its architecture.
This project was awarded by the \href{http://itu.int/en/ITU-T/challenges/pages/iptv.aspx}{2nd ITU IPTV Application Challenge} and become the bases for another awarded project that I worked called Brasil4D.

The \href{http://www.ebc.com.br/brasil-4d}{Brasil4D} was funded by the \href{https://www.worldbank.org/}{World Bank} and focused on Ginga social impact.
This project performed transmission and digital reception of interactive video applications (using Ginga technology) to 100 low-income families.
Those applications presented education and health content, besides labor opportunities.
The Brazilian Broadcast Community SET awarded Brasil4D as \href{http://set.org.br/artigos/ed137/137_revistadaset_70.pdf}{Best interactivity solution}.
It also received awards from the \href{https://www.premiotv.com/es/ganadores-es/ganadores-2013-es}{La Cumbre TV Abierta} and \href{https://programafrida.net/archivos/project/brasil-4d}{FRIDA Program} because its innovative technological solutions and social impact.
We describe this impact in a  \href{http://documents.worldbank.org/curated/en/232621468230956108/pdf/809560WP0PORTU0Box0379824B00PUBLIC0.pdf}{report} from the World Bank.

At \textbf{Rio de Janeiro}, I acquired a Ph.D.~(2017) degree from the \href{http://www.inf.puc-rio.br/}{PUC-Rio} and acted as Research Assistant at \href{http://telemidia.puc-rio.br/}{TeleMidia Lab}.
I contributed to new versions of the Ginga standard.
First, I work on \href{https://www.abntcatalogo.com.br/norma.aspx?
  ID=361857#}{incorporate features and requirements from the Brasil4D project} into the standard.
And more recently, we target features \href{http://www.freepatentsonline.com/y2016/0234533.html}{IBB} (Integrated Broadcast Broadband) context.
We prototyped these evolutions into the  \href{https://github.com/TeleMidia/ginga}{Ginga ITU reference implementation}, which is consulted by universities and companies.
On my Ph.D., professor \href{https://www.researchgate.net/profile/Luiz_Fernando_Soares}{Luiz Fernando Gomes Soares} (\textit{in memmorian}) initially advised me and then followed by  \href{https://www.linkedin.com/in/simonedjb/}{Simone Barbosa}.
Given the popularization of systems with multimodal interactions, such as personal assistants (e.g., Alexa, Siri, Cortana, and Google Assistant), we aimed at involving Ginga from a video-oriented to a multimodal-oriented system.
We publish this evolution in a recognized \href{https://link.springer.com/article/10.1007\%2Fs11042-016-3846-8}{Multimedia Journal}.

Recently on TeleMidia Lab, I worked with Blockchain and Machine Learning.
The \href{http://wrnp.rnp.br/sites/wrnp2017/files/02_wrnp2017_poster_gt-sap_design.pdf}{GT-RAP} stands for Registration, Authentication, and Digital Document Preservation.
This project was funded by RNP and aimed at using blockchain technology to public document registration and authentication.
In Machine Learning, I worked in AML and VideoMR.
AML stands for Anti Money Laundry and was funded by \href{https://www.btgpactual.com/}{BTG
Bank}.
This project developed a system that uses Machine Learning to understand money transactions and detect money laundry patterns.
\href{https://www.rnp.br/en/rnp-and-microsoft-challenge-artificial-intelligence}{VideoMR} (Video Mature Rating) was funded by RNP and Microsoft.
This project aims at evaluating and developing \textit{Deep Learning} methods for detecting inappropriate content in video scenes.

\vspace{1em}
\raggedright
Sincerely, \myname

\end{document}